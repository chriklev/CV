
% =============================================================================
% PERSONLIG INFORMASJON
% =============================================================================
\newcommand{\myname}{Christoffer Kleven Berg}
\newcommand{\myaddress}{Nydalen Allé 5, 0484, Oslo}
\newcommand{\myphone}{+47 944 85 055}
\newcommand{\myemail}{cklevenb@gmail.com}

% Valgfritt: LinkedIn, GitHub, Portfolio
% \newcommand{\mylinkedin}{linkedin.com/in/yourprofile}
% \newcommand{\mygithub}{github.com/yourusername}
% \newcommand{\mywebsite}{yourwebsite.com}

% =============================================================================
% PROFESJONELT SAMMENDRAG
% =============================================================================
\newcommand{\mysummary}{%
    Data Engineer med 3+ års erfaring i å bygge robuste dataflyter for rapporterings og analyseformål.
    Sterkt grunnlag i matematikk, statistikk og informatikk fra studiet, kombinert med praktisk erfaring innen skybaserte
    dataplattformer (Databricks, BigQuery, Fabric), verktøy for datatransformasjoner (DBT, Dataform) og ETL-utvikling.
    Jeg lærer raskt og setter meg fort inn i nye teknologier, og trives i fleksible konsulentmiljøer med komplekse utfordringer.
    Jeg jobber selvstendig og liker å ta mye ansvar i prosjekt/leveranser.%
}

% =============================================================================
% ERFARING
% =============================================================================
\newcommand{\experiencesection}{%
    \section{Arbeidserfaring}

    \cvrole{Axaz AS}{Oslo, Norge}{IT-konsulent - Data Engineer}{Aug 2022--Nå}
    \begin{itemize}
        \item Jobber som konsulent i Data \& Innsikt-avdelingen, utvikler dataløsninger på tvers av flere kundeprosjekter
        \item Designer og implementerer ETL-prosesser og dataflyter ved bruk av ulike dataplattformer inkludert Databricks, BigQuery og Microsoft Fabric
        \item Bygger datatransformasjoner med DBT, Dataform og Spark-baserte rammeverk for analyse- og rapporteringsløsninger
        \item Administrerer infrastruktur og arbeidsflyter ved bruk av skyplattformer (Azure, GCP) og infrastruktur som kode (Terraform, bicep, databricks asset bundle)
        \item Jobber med data på tvers av flere bransjer og domener. Mye relatert til finans, salg/innkjøp, timeregistrering/fravær.
        \item Utvikler og drifter integrasjonsløsninger for Integrasjonsavdelingen
        \item Jobber selvstendig og tett med kundene i prosjekter
        \item Bidrar til forretningsutvikling og salg/demoarbeid når det er tid
    \end{itemize}

    % \begin{itemize}
    %     \item Konsulent i Data \& Innsikt-avdelingen, utvikler dataløsninger på tvers av flere kundeprosjekter
    %     \item Designer og implementerer ETL-prosesser og dataflyter ved bruk av ulike dataplattformer (Databricks, BigQuery, Microsoft Fabric)
    %     \item Bygger datatransformasjoner med DBT, Dataform og Spark for analyse- og rapporteringsløsninger
    %     \item Administrerer datainfrastruktur med infrastruktur som kode (Terraform, Bicep, Databricks Asset Bundles)
    %     \item Jobber selvstendig og tett med kunder, med ansvar for design, implementering og leveranse
    % \end{itemize}
    % 
    % \textbf{Nøkkelprosjekter:}
    % 
    % \textit{Norli AS - Migrering av datavarehusløsning}
    % \begin{itemize}
    %     \item Migrerte datavarehus fra on-prem Oracle ERP til Azure Databricks og Power BI
    %     \item Ansvarlig for design og oppsett av arkitektur, samt implementering av dataplatform med komplekse datakilder (on-prem/cloud databaser, API-er, filservere, service bus)
    %     \item Bygget ETL-prosesser og datatransformasjoner med DBT, implementert tilgangsstyring og tilgjengeliggjøring av data
    %     \item Leverte kundeservice-dashboard som bidro til 60\% raskere behandling av kundehenvendelser
    %     \item Migrerer 50+ rapportsider for finans, regnskap og salg som vil brukes av over 200 sluttbrukere
    % \end{itemize}

    \cvrole{Kiwi Nordstrand}{Oslo, Norge}{Låseansvarlig}{Aug 2021--Okt 2021}

    \cvrole{Kiwi Nordstrand}{Oslo, Norge}{Butikkmedarbeider 30\%}{Aug 2018--Jul 2021}

}

% =============================================================================
% UTDANNING
% =============================================================================
\newcommand{\educationsection}{%
    \section{Utdanning}

    \cveducation{Universitetet i Oslo}{Oslo, Norge}{Master i Data Science (Ett år fullført)}{Aug 2020--Jun 2021}

    \cveducation{Universitetet i Oslo}{Oslo, Norge}{Bachelor i Matematikk med Informatikk}{Aug 2016--Jun 2020}
}

% =============================================================================
% FERDIGHETER
% =============================================================================
\newcommand{\skillssection}{%

    \section{Ferdigheter}

    \cvskillcategory{Språk}{Python, SQL}

    \cvskillcategory{Skyplattformer}{Azure, GCP}

    \cvskillcategory{Dataplattformer}{Databricks, BigQuery, Fabric}

    \cvskillcategory{ETL/Integrasjon/Orchestration}{Data Factory, Boomi, Databricks Jobs}

    \cvskillcategory{Databehandling}{DBT, Dataform}

    \cvskillcategory{DevOps \& Infrastruktur}{Git, Linux, Docker, Terraform}

    % Alternativ: Enkelt linje ferdighetsliste
    % \cvskills{Python, SQL, Apache Spark, GCP, Docker, Git}
}

% =============================================================================
% SERTIFISERINGER
% =============================================================================
\newcommand{\certificationssection}{%
    \section{Sertifiseringer}

    \begin{certlist}{Skysertifiseringer}
        \item Google Associate Cloud Engineer (Jul 2023)
        \item Google Cloud Digital Leader (Feb 2023)
        \item Microsoft Azure Security Fundamentals (Mar 2025)
        \item Microsoft Azure Data Fundamentals (Jul 2024)
        \item Microsoft Azure Fundamentals (Jul 2023)
    \end{certlist}

    \begin{certlist}{Boomi}
        \item Professional API Management (Sep 2022)
        \item Professional Developer (Aug 2022)
        \item Associate Master Data Hub (Aug 2022)
    \end{certlist}

    \begin{certlist}{Sikkerhet}
        \item NSM grunnprinsipper for IKT-sikkerhet (Okt 2022)
    \end{certlist}
}

% =============================================================================
% PROSJEKTER (Valgfritt)
% =============================================================================
\newcommand{\projectssection}{%
    % Fjern kommentar og tilpass hvis du vil inkludere prosjekter
    % \section{Prosjekter}
    %
    % \cvrole{Prosjektnavn}{}{Personlig/Åpen kildekode}{}
    % \begin{itemize}
    %     \item Kort beskrivelse av prosjektet og dine bidrag
    %     \item Teknologier brukt og påvirkning/resultater oppnådd
    % \end{itemize}
}
